\documentclass[a4paper,12pt]{article}

% Import necessary packages
\usepackage{xcolor} 
\usepackage{listings} 
\usepackage{geometry} 
\usepackage{tcolorbox} 
\usepackage{multicol} % For multi-column support

% Set page margins (narrower margins)
\geometry{top=0.75in, bottom=0.75in, left=0.5in, right=0.5in}

% Define code style with very small font size and line wrapping
\lstdefinestyle{mystyle}{
    backgroundcolor=\color{gray!10},   % Light gray background for the code
    basicstyle=\ttfamily\tiny,         % Font size for code
    breaklines=true,                   
    captionpos=b,                    
    numbers=left,                    
    numberstyle=\tiny\color{gray},     % Line numbers style
    keywordstyle=\color{blue},         % Keywords color (blue)
    commentstyle=\color{magenta},      % Comments color (magenta)
    stringstyle=\color{red},           % Strings color (red)
    showstringspaces=false
}

% Set style for code listings
\lstset{style=mystyle}

% Define a custom tcolorbox for code
% \newtcolorbox[auto counter, number within=section]{mycode}[2][]{colframe=blue!50!white, colback=blue!5, 
% coltitle=black, fonttitle=\bfseries, title=Code Example~\thetcbcounter: #2,#1, width=\linewidth}
\newtcolorbox[auto counter, number within=section]{mycode}[2][]{colframe=blue!50!white, colback=blue!5, 
coltitle=black, fonttitle=\small\bfseries, title=#2,#1, width=\linewidth}



\title{Code Snippet Repository}
\author{Your Name}
\date{\today}

\begin{document}

\maketitle

% Use multi-column layout with 2 columns
\begin{multicols}{2}

\section{Python Code Snippets}

\subsection{Max Area of Island}
\begin{mycode}[label={lst:max-area-of-island}]{Max area of islands}
\begin{lstlisting}[language=Python]
from collections import deque

class Solution:
    def maxAreaOfIsland(self, grid: List[List[int]]) -> int:
        m = len(grid)
        n = len(grid[0])
        count = 0
        self.max_area = -float('inf')

        def bfs(i,j):
            queue = deque()
            queue.append((i,j))
            grid[i][j] = 0 # marked as visited
            area = 1

            while queue:
                ii, jj = queue.popleft()
                for di, dj in [(-1, 0), (1, 0), (0, 1), (0, -1)]:
                    ni, nj = ii + di, jj + dj
                    if 0 <= ni < m and 0 <= nj < n and grid[ni][nj] == 1:
                        queue.append((ni, nj))
                        grid[ni][nj] = 0 # marked as visited
                        area += 1
            self.max_area = max(self.max_area, area)

        
        for i in range(m):
            for j in range(n):
                if grid[i][j] == 1:
                    bfs(i,j)
                    count += 1
        return max(self.max_area, 0)
\end{lstlisting}
\end{mycode}

\subsection{Koko eating bananas}
\begin{mycode}[label={lst:koko-eating-bananas}]{Binary Search}
\begin{lstlisting}[language=Python]
import math
class Solution:
    def minEatingSpeed(self, piles: List[int], h: int) -> int:
        left, right = 1, max(piles)
        while left < right:
            mid = (left + right) // 2
            # Calculate round up ceil() number of hours
            # hours = sum((pile + mid - 1) // mid for pile in piles)
            hours = sum(math.ceil(pile / mid) for pile in piles)
            if hours > h:
                left = mid + 1
            else:
                right = mid
        return left
\end{lstlisting}
\end{mycode}

\columnbreak

\subsection{Two-sum}

\begin{mycode}[label={lst:two-sum}]{Two Sum}
\begin{lstlisting}[language=Python]
import math
class Solution:
    def minEatingSpeed(self, piles: List[int], h: int) -> int:
        left, right = 1, max(piles)
        while left < right:
            mid = (left + right) // 2
            # Calculate round up ceil() number of hours
            # hours = sum((pile + mid - 1) // mid for pile in piles)
            hours = sum(math.ceil(pile / mid) for pile in piles)
            if hours > h:
                left = mid + 1
            else:
                right = mid
        return left
\end{lstlisting}
\end{mycode}


\end{multicols}

\end{document}
